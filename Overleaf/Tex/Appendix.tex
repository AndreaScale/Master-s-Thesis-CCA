\section*{Appendix}
\addcontentsline{toc}{section}{Appendix}

Let us recall some basic facts of order theory.

\begin{definition}
    Let $X\neq\emptyset$, a binary relation $R$ on $X$ is a subset of $X\times X$. We write 

    \begin{equation}
        xRy \text{ if } \{x,y\}\in R.
    \end{equation}
\end{definition}

We define some basic properties of a binary relation.

\begin{definition}
    Let $R$ be a binary relation on $X$ non-empty. We define

\begin{itemize}
        \item Reflexivity: $\forall x\in X:\, xRx$.
        \item Irreflexivity: $\nexists x\in X:\, xRx$.
        \item Antisymmetry: $xRy,yRx \Rightarrow x=y$.
        \item Symmetry: $xRy \Rightarrow yRx$.
        \item Transitivity: $xRy,yRz \Rightarrow xRz$.
        \item Completness: $\forall x,y\in X \Rightarrow xRy\text{ or }yRx$.
    \end{itemize}
\end{definition}

Typically, a preorder is a binary relation that satisfies reflexivity, transitivity, and symmetry, and a set with a preorder is called a poset. A linear order is a complete preorder, and a set with a linear order is called a loset. 
Most of order theory is developed on these concepts. Instead, we use antisymmetry instead of symmetry, and irreflexivity instead of reflexivity. 


Let us now define the transitive closure of a binary relation.

\begin{definition}
    Let $R$ be a binary relation on $X$ non-empty. The transitive closure of $R$ is the smallest transitive binary relation that contains $R$. It is denoted with $T(R)$.
\end{definition}

Inclusion of binary relation is intended in the canonical set inclusion, that is given $R,S$ binary relation we say that $R$ is included in $S$ if

\[xRy \Rightarrow xSy,\]

or equivalently $R\subseteq S$.

\begin{proposition}
    Every binary relation $R$ has a transitive closure $T(R)$.

    \begin{proof}
        The binary relation $S=X\times X$ is transitive and contains $R$. Furthermore, the transitivity of a binary relation is closed under intersection. Indeed, if $V,W$ are transitive binary relations then $U=V\cap W$ is transitive: if $xUy,yUz$ then $xVy,yVz$ and the same thing for $W$, then by transitivity of $V,W$ we have $xUz$ and $xVz$, that is $xUz$. 
    \end{proof}
\end{proposition}

We define the extension of a binary relation.

\begin{definition}
    Let $R$ be a binary relation on $X$ non-empty. An extension $R^{\ast}$ of $R$ is a binary relation that contains $R$ and preserves the properties of $R$.  
\end{definition}

We now arrive at the fundamental result used in the jungle policies section.

\begin{theorem*}
    Let $X$ non-empty and $R$ irreflexive, transitive, and antisymmetric binary relation, then there exists a complete extension of $R$.
\end{theorem*}

Before proving the theorem we clarify an unproved result that we will use, the Hausdorff Maximum principle. Let us assume the Axiom of Choice.

\begin{axiom}
    Let $\mathcal{A}$ be an arbitrary collection of non-empty sets. Then there exists a function $f:\mathcal{A} \rightarrow \cup_{A\in\mathcal{A}}A$ such that

    \[f(A)\in A,\,\forall A\in\mathcal{A}.\]
\end{axiom}

It can be proven that the axiom of choice is equivalent to the Hausdorff's Maximum principle. A proof of necessity can be found in Rudin \ref{rudin1976principles} (p. 395).

\begin{axiom}
    % Every set $X$ with an irreflexive, transitive and antisymmetric binary relation admits a maximal subset $Y$ with an irreflexive, transitive and antisymmetric binary relation.
    There exists an $\supseteq$-maximal loset in every poset.
\end{axiom}

Maximality is intended as no other subset with such a binary relation contains $Y$ and is in $X$. Thanks to this result we can prove the theorem, assuming the Axiom of Chioce.

\begin{proof}
    Let $T_X$ be the set of all extensions of $R$. By definition of set inclusion $\supseteq$ then $(T_X,\supseteq)$ is a poset. Let $(A,\supseteq)$ be a maximal loset in $T_X$. 
    Let us define $R^{\ast}=\cup_{S\in A}S$, that is the union of all binary relations in $A$. Since $A$ is a loset, then union of all its element is its maximal element, therefore $R^{\ast}$ 
    is an element of $A$, and therefore an extension of $R$. We now show that $R^{\ast}$ is complete. If it wasn't there'd be at least two elements $x,y\in X$ unranked by $R^{\ast}$. 
    Let us define $R'=T\left(R^{\ast}\cup\{x,y\}\right)$. It is an extension of $R$\footnote{It can be directly proven.} and strictly contains $R^{\ast}$, which contradicts the maximality of $A$.
\end{proof}