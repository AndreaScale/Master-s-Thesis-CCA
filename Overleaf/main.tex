\documentclass[12pt,a4paper]{article}
\usepackage[a4paper, margin=1.25in]{geometry}
\usepackage[utf8]{inputenc}
\usepackage[english]{babel}
\selectlanguage{english}
\usepackage{amsmath,amsthm,amssymb,fge,mathrsfs}
\usepackage{mathtools}
\usepackage[hidelinks, colorlinks = true, linkcolor = gray, urlcolor = magenta, menucolor = black, citecolor = gray]{hyperref}
\usepackage{makeidx} %per fare l'indice
\usepackage{faktor} %per i quozienti
\usepackage{csquotes}
%tikz (pacchetto per i disegni)
\usepackage{tikz}
\usetikzlibrary{arrows}
\usetikzlibrary{tikzmark}
\usetikzlibrary{calc} %per poter fare i calcoli
\usetikzlibrary{arrows.meta} %per usare veri tipi di frecce
\usetikzlibrary{calc,patterns,angles,quotes} %per disegnare gli angoli
\usetikzlibrary{decorations} %per i grafici orientati (con sopra le frecce)
\usetikzlibrary{decorations.markings}
\usetikzlibrary{backgrounds} %per il colore sullo sfondo delle immagini
\usetikzlibrary{shapes.geometric} %per i triangoli negli alberi

\usepackage{ stmaryrd }

\usepackage{pgfplots}
\pgfplotsset{compat=newest}
\usepgfplotslibrary{fillbetween} %per colorare le aree comprese tra due grafici

\usepackage{caption} %per la caption sotto alle immagini
\usepackage[tracking=true]{microtype} %per avvicinare il testo
\usepackage{enumitem} %per modificare agilmente gli elenchi
\usepackage{esvect} %per le freccette di vettore sui caratteri
\usepackage{stmaryrd} %fulmine per il simbolo dell'assurdo
\usepackage{relsize} %per fare i simboli più grandi
\usepackage{mathabx} %per il simbolo di dotminus
\usepackage{esint} %per l'integrale tagliato (fint)

\usepackage[customcolors]{hf-tikz} %per evidenziare pezzi di matrice

\usepackage{verbatim} %per commentare un blocco

\usepackage{multirow} %per la tabella
\usepackage{array} %per lo spazio nella tabella

%Libreria per gli alberi
\usepackage[linguistics]{forest}

%Librerie aggiunte
\usepackage{multicol} %per gli elenchi puntati su più colonne/righe
\usepackage{multirow}

% \write18
%pacchetto per il prodotto interno
\usepackage{physics}

%Libreria per i subfiles

\usepackage{subfiles} %meglio caricarla per ultima

\usepackage{biblatex}
\addbibresource{biblio.bib}
%usiamo paragraph come nuova sottosottosottosezione 
\makeatletter
\renewcommand\paragraph{\@startsection{paragraph}{4}{\z@}%
            {-2.5ex\@plus -1ex \@minus -.25ex}%
            {1.25ex \@plus .25ex}%
            {\normalfont\normalsize\bfseries}}
\makeatother
\setcounter{secnumdepth}{4} % how many sectioning levels to assign numbers to
\setcounter{tocdepth}{4}    % how many sectioning levels to show in ToC

\newtheorem{theorem}{Theorem}
\numberwithin{theorem}{section}
\newtheorem*{theorem*}{Theorem}
\newtheorem{corollary}[theorem]{Corollary}
\newtheorem{proposition}[theorem]{Proposition}
\newtheorem{lemma}[theorem]{Lemma}
\newtheorem*{lemma*}{Lemma}

\newtheorem{definition}{Definition}
\newtheorem{axiom}{Axiom}
\numberwithin{definition}{section}

\newtheorem*{remark}{Remark}

\newtheorem{example}{Example}
\numberwithin{example}{section}

\newtheorem{exercise}{Exercise}
\numberwithin{exercise}{section}

\renewcommand{\proofname}{\proof}

\renewcommand{\thefootnote}{[\arabic{footnote}]}% Modify footnote globally

%%%simboli speciali
\newcommand{\N}{\mathbb{N}}
\newcommand{\Z}{\mathbb{Z}}
\newcommand{\Q}{\mathbb{Q}}
\newcommand{\R}{\mathbb{R}}
\newcommand{\C}{\mathbb{C}}

\newcommand{\CO}{\mathcal{C}}

\DeclareMathOperator{\Dom}{Dom}
\DeclareMathOperator{\Ima}{Im}

\DeclareMathOperator{\mcd}{MCD}
\DeclareMathOperator{\mcm}{mcm}
\DeclareMathOperator{\supp}{supp}
\DeclareMathOperator{\spanlin}{span}

%\DeclarePairedDelimiter\abs{\lvert}{\rvert}
%\DeclarePairedDelimiter\norm{\lVert}{\rVert}

%Necessario per far si che la dimensione del valore assoluto e della norma si adatti all'aromento passato
%\makeatletter
%\let\oldabs\abs
%\def\abs{\@ifstar{\oldabs}{\oldabs*}}
%
%\let\oldnorm\norm
%\def\norm{\@ifstar{\oldnorm}{\oldnorm*}}
%\makeatother

%rinomino il comando per il prodotto interno
\let\pint\braket
%comando per il prodotto scalare
\newcommand{\pscal}[2]{\left\langle #1, #2 \right\rangle}

%Spaziatura per i quantificatori
\let\oldforall\forall
\renewcommand{\forall}{\; \oldforall \;}
\let\oldexists\exists
\renewcommand{\exists}{\; \oldexists \;}

%freccia per la convergenza debole
\newcommand{\longrightharpoonup}{\xrightharpoonup{\phantom{AB}}}

\newcommand{\warrow}{\xrightharpoonup{\:\, w \:\, }}
\newcommand{\sarrow}{\overset{s}{\longrightarrow}}
\newcommand{\wastarrow}{\xrightharpoonup{ w^* }}

\setlength{\parindent}{0in} 
\title{Bananas in a Civilized Jungle}
\author{Andrea Scalenghe}
\date{\today}

\begin{document}

\thispagestyle{empty}

\centerline {\Large{\textsc{COLLEGIO CARLO ALBERTO}}}
\vskip 20 pt

\centerline {\Large{\textsc ALLIEVI HONORS PROGRAMM}}

\vskip 20 pt

\centerline {{\textsc Economics, Statistics and Applied Mathematics Track}}

\vskip 20 pt

\centerline {\Large{\textsc Master's Degree in Economics}}
\vskip 60 pt





%\begin{tabular}{ccc}
\centerline {\includegraphics[width=5cm]{cca.png}}
%   \end{tabular}

\vskip 1.2cm

\centerline {\normalsize {Master's Thesis}} 

\vskip 0.7cm

\centerline {\Large {\bf Bananas in a Civilized Jungle} }
% \vskip 0.3cm
% \centerline {\Large {\bf Optimal Control and Dynamic Programming}}

\vskip 1.7cm

\noindent Supervisor: Dino Gerardi
\hfill  {Candidate: Andrea Scalenghe}

\vskip 2.7cm


\centerline{Academic Year 2023-2024}

\newpage
{
\hypersetup{linkcolor = black}
\tableofcontents
\addtocontents{toc}{~\hfill\textbf{Page}\par}
}
\newpage

\section*{Introduction}
\addcontentsline{toc}{section}{Introduction}

In traditional economic models, power relations are often absent, with assumptions typically focusing on agents interacting in a market where resources are allocated based purely on preferences and market dynamics. However, real-world scenarios frequently involve significant power disparities, influencing how resources are distributed. The "jungle" framework addresses this gap by integrating power structures into economic analysis, providing a different understanding of resource allocation.

The jungle framework, as introduced by Piccione and Rubinstein (2007) \cite[PR]{P-R}, examines resource distribution among agents influenced by both their preferences and their hierarchical positions. This dual influence creates a dynamic environment where conventional notions of equilibrium and welfare are reevaluated. Rubinstein and Yildiz (2022) \cite[RY]{RY} enlarge this setting by introducing the concept of civilization. Civilization adds a layer of complexity and realism by incorporating the notion of language. Agents within this framework strive to follow their preferences through brute force while primarily governed by the common language they share. Their social position is determined through a blend of meritocratic and hierarchical principles. The former represented by the language, a collection of orderings over the agents, while the latter by a unique strict order.

This dissertation aims at further enlarge the civilized jungle setting by accounting for divisible commodities. The original structure of the civilized jungle framework follows a clear sequence: first defining the setting, then defining the equilibrium, and finally studying the welfare theorems. We will follow the same scheme. In particular, the structure of this dissertation is organized as follows:

The first section introduces the original civilized jungle framework, offering a comparison between jungle equilibria and civilized equilibria. This comparison highlights the fundamental differences in resource allocation mechanisms under varying equilibrium concepts, establishing a foundation for further exploration.

The second section extends the model to incorporate divisible commodities. We define the new equilibrium concepts and present key examples to illustrate the impact of this extension. This section also compares equilibrium concepts with and without civilization, revealing significant changes in allocation processes and economic outcomes.

The third section explores the welfare theorems within the civilized jungle context, analyzing their behavior and relevance. This examination is crucial for understanding the implications of efficiency and equity in resource allocation within hierarchical structures.

The fourth section addresses the broader issue arising from the previous segment, when can an allocation be supported as an equilibrium, disregarding its efficiency. To answer this question a deeper understanding of the equilibrium concept allows us to spot a sufficient condition for equilibrium. Here language is seen as a tool on the hand of jungle institutions, through which they can implement certain kind of allocations as equilibria, together with a suitable power relation.     

% The fourth section addresses the second welfare theorem, investigating the conditions under which an allocation can be sustained as a civilized equilibrium. While not directly answering the role of Pareto efficiency, this section presents the dissertation's original contribution by demonstrating how civilization can effectively implement civilized equilibria.

Finally, the appendix provides a formal treatment of the mathematical objects utilized in the fourth section. 

In the final remarks we point out the limits of this theory, its strengths and possible area of new studies. In particular, we try to throw light on the intrinsic limitations of the jungle setting and where no space of development of the theory may take place. Nevertheless, we also emphasize in which scenarios this mathematical modelization of reality better describes it.   


% We leave unproved the equivalence of the Axiom of Choice and the Hausdorff's Maximum principle since it goes beyond the scope of this dissertation.


% This dissertation aims to extend the setting in which the civilized jungle concept is presented. I will describe the basic features of a civilized jungle as presented by A. Rubinstein and K. Yildiz \cite{RY} and show some of their results. Then I formally present a brief review of the welfare theorems, which are a core topic of the essay. Then I enlarge the setting. At first, I analyze a civilized jungle with divisible goods, which is done by introducing consumption bundles for the commodities and consumption set for the agents. This first part has fundamental insights from the paper by M. Piccione and A. Rubinstein \cite{P-R}. Secondly I introduce production. My ultimate goal is to study an analogous of the second welfare theorem for a civilized jungle with divisible goods and production. 

% It is important to point out the limitation of a representative firm \cite{Growth_theory-Acemoglu}(Theorem 5.4).

\newpage

\section{Civilized Jungle}

Rubinstein and Yildiz \cite{RY} present a civilized jungle as a tuple $\langle N,X,(\succeq^i)_{i\in N}, \trianglerighteq, \mathcal{L}\rangle$ where $N$ are the agents, $X$ are the $N$ object, $(\succeq^i)_{i\in N}$ are the preferences of each agents over the objects, $\triangleright$ is a strict and complete ordering over $N$ which represents the power relation between the agents and $\mathcal{L}=\{\geq_{\lambda}\}_{\lambda\in\Lambda}$ is a set of complete and transitive (possibly not antisymmetric) binary relations which stand for different criteria that rank agents. 

Since commodities are indivisible, a distribution of the resources over the agents is represented by an assignment $\textbf{x}$ that is a map that maps every agent to an object, that is $\textbf{x}:N\rightarrow X:i\mapsto x^i$. We will use the notation $\textbf{x}=(x^1,\dots,x^2)$. How do we define an equilibrium? The basic, yet exhaustively informative idea is that Given an assignment, every agent has to be the strongest among those who envy her and can justify themselves. How can someone justify herself among a group of agents? We say that agent $i\in N$ is justified by $\geq_{\lambda}$ over $I\subseteq N$ if $i\geq_{\lambda} j\,\forall j\in I$, then:

\begin{center}
$i$ is justified among $I$ if $\exists\lambda\in\Lambda$ such that she is justified by that criteria over $I$.
\end{center}

That is, if you are the best in a group for at least one criterion then you are justified in that group. We want to highlight that the justification concept is exclusively meritocratic, and therefore far from reality. Nevertheless, this is what we mean by bringing \textit{civilization} into the jungle. 
Notationally, we define $J_{\mathcal{L}}(I)$ to be the set of justifiable agents in $I$. Rigorously, let $\tilde{J}_{\mathcal{L}}:\mathcal{P}(N)\rightarrow\mathcal{P}(N)$ such that

\[\Tilde{J_{\mathcal{L}}}(I)=\left\{i\in N\,|\, \exists\lambda\in\Lambda\text{ s.t. }i>^{\lambda}j\,\forall j\in I\right\},\]

then the definition of justified agents in a group is given by

\begin{equation}\label{eqn: J_L}
    J_{\mathcal{L}}(I)=\Tilde{J_{\mathcal{L}}}(I)\cap I.
\end{equation}

We restrain the justification only to those who are in the group since we will use this notion just for those who envy or dream about someone, as we will see. Thus, those who are not envious are not interested in being justified. 

The last core concept, as mentioned, is \textit{envy}. Trivially, an agent $i$ envies another agent $j$ if $x^j\succeq^ix^i$. Here the idea is clear, if I like better your commodity than mine, I envy you. We define $E(\textbf{x},i)$ as the set of agents that envies $i$, given the assignment $\textbf{x}$. 

We are now able to define the equilibrium concept in a civilized jungle.

\begin{definition}
    An assignment $\textbf{x}$ is a \textbf{civilized equilibrium} if $\forall i\in N$ then $i\in J_{\mathcal{L}}(E(\textbf{x},i))$ and:
    \[i\triangleright j \quad \forall j\in J_{\mathcal{L}}(E(\textbf{x},i)) \]
\end{definition}

A civilized equilibrium requires an assignment to have a fundamental feature; it must be the case that each agent is justified among those who envy her, if not he couldn't even fight for her good, since civilization would prevent her to do so. Only after, when the civilization barrier is overcome, she must be the strongest among those who want, and are entitled, to fight her.  

A civilized equilibrium is also called a C equilibrium. The existence of such an equilibrium will be discussed in different instances, and it will help us understand how civilization embroils jungle's nature. Interesting languages are those that partition agents into two indifferent sets, for each criterion. Such languages are called dichotomous languages.
%  Such an equilibrium always exists, although it is not very interesting being just a serial dictatorship according to the power relation. We are referring to the simplest equilibrium  

\begin{definition}\label{def: dichotomous language}
    A language $\mathcal{L}$ is dichotomous if for each criterion $\lambda\in\Lambda$ there exist $i,j\in N$ such that $i>_{\lambda}j$ and $\forall k\in N$ either $k=_{\lambda}i$ or $k=_{\lambda}j$.
\end{definition}

Given a dichotomous language $\mathcal{L}$ we can therefore partition $N$ by $\sim_{\lambda}$ for all $\lambda\in\Lambda$, where:

\begin{equation}\label{eqn: def sim}
    i\sim_{\lambda} j \Leftrightarrow i=_{\lambda}j.
\end{equation}

By doing so we get $N/\sim_{\lambda}=\{0,1\}$. Another handy way of representing a dichotomous language is to define a set of properties $\phi^i$ for each agent $i\in N$. These sets are defined to be the collection of criteria with respect to which the agent is in the upper equivalence class defined by \cref{eqn: def sim}. In economic terms, a dichotomous language provides a distinction into strong and weak agents for each criterion. 


\begin{example}[Collegio Carlo Alberto]\label{Example: CCA}
    Let the Collegio Carlo Alberto be a Jungle, civilized. The agents are the Professors and the Allievi. The commodities are different quality of wines at a buffet in the common room: there is only one glass for each quality ranging from a "Barbaresco di Gaja" all the way to a bottle of wine in carton of a discount. Of course, everyone has the same preference relation: better quality is better. We assume that there the power relation is defined by each agent's capability of winning a buffet race. Nevertheless, one can exercise her power only if she is justified by the one criterion that governs the Collegio: being a scholar. Does this setting admit a C equilibrium? We formalize by letting $N$ finite, $X$ finite of same cardinality of $N$, where each number corresponds to a different wine and $x\succeq^i y$ if and only if $x\geq y$, where $\leq$ is the usual ordering over the natural numbers for all $i\in N$. We define the power relation $\triangleright$ and the language $\Lambda=\{\lambda\}$ is a singleton and $i>_{\lambda}j$ if and only if $i$ is a Professor and $j$ an Allievo. The only civilized equilibrium is straightforward: the serial dictatorships over the two equivalence classes induced by $\sim_{\lambda}$. Therefore, the fastest and mostly skilled Professor will drink the Gaja, and all the others will follow, based on the power criteria that govern their world. Only after, when so-called civilization has played its role, will the Allievi fight for the remaining glasses, based on the power relation that reigns among them. 

    % In general, if $\mathcal{L}$ is a singleton whose element is not a strict ordering,    as before, one can define the C equilibrium running serial dictatorships over all the equivalence classes induced by $\sim$.
\end{example}

% \begin{example}
%     R-Y\cite{RY} show that if we assume the same preferences $a_1\succ\dots\succ a_n$ for all agents and $\mathcal{L}$ contains at least one strict ordering, then the unique C equilibrium is given by assigning $i_l$ to $a_l$. 

    

%     In the most general context of a dichotomous language, where there are multiple criteria, 
% \end{example}
% \textit{THINK BETTER
% \begin{example}
%     Lower cost of labor in developing countries: agents are firms, commodities are salaries (all negative commodities, preference relation is usual over real numbers), power is money and general power of the firm, language are state laws. 
% \end{example}}

% \color{red}{STOCHASTIC ORDERING}

\color{black}{A language truly brings civilization to a jungle if it has more than one criterion.}\color{black}{} In fact as \cite[RY]{RY} show, if $\mathcal{L}$ consists of just one strict ordering $\geq$, then the unique "civilized" equilibrium is a serial dictatorship according to $\geq$. Then the C equilibrium does not depend on the power relation over the agents\color{black}{ and the civilized jungle is just a normal jungle with another power relation.}  \color{black} 

This example highlights a technique that can be used in different instances when searching for a civilized equilibrium. Indeed, it is used to prove the following result.

\begin{proposition}\label{prop: strict ordering}
    If $\mathcal{L}$ consists of just one strict ordering $\geq$, then the unique civilized equilibrium is a serial dictatorship according to $\geq$.
    \begin{proof}
        It is an equilibrium since the strongest with respect to $\geq$ is going to be the only one justified among those who envy about her. We rule her out. Proceeding iteratively we get the equilibrium definition. This argument almost shows uniqueness of the equilibrium. Let us consider another equilibrium. In this equilibrium there must be some agent $j$ dreaming about $i$ and being stronger, civilization wise, than her and weaker to no one in the set of those who dream about $i$. If so, she would be the only one justified, absurd.
    \end{proof}
\end{proposition}

The reasoning used in Proposition \ref{prop: strict ordering} allows us to state also the following property.

\begin{proposition}\label{prop: L singleton}
    If $\mathcal{L}$ is a singleton then the serial dictatorships over all the equivalence classes induced by equivalent power is a C equilibrium.  
\end{proposition}

We can think of an uncivilized jungle. Piccione and Rubinstein \cite[PR]{P-R} defined the concept of jungle, as a civilized jungle without a language and a related equilibrium.

\begin{definition}
    An assignment is a \textbf{jungle equilibrium} if: \[\forall i \in N \, \forall j \in E(\textbf{x},i) \text{ then } i\triangleright j\]
\end{definition}

Therefore, the unique jungle equilibrium is obtained through a serial dictatorship governed by the power relation. 

\begin{proposition}
    Let $\langle N,X,(\succeq^i)_{i\in N}, \trianglerighteq\rangle$ be a jungle, a jungle equilibrium is given by the assignment obtained through the serial dictatorship ruled by $\trianglerighteq$.

    \begin{proof}
        We define a serial dictatorship ruled by $\trianglerighteq$ recursively as $x^1=\max_{\succeq^1}X$ and:

        \[x^i=\max_{\succeq^i}X\setminus\{x^1,\dots,x^{i-1}\}\]

        Where $\max_{\succeq^j}Y$ stands for maximize $Y$ with respect to the preference relation $\succeq^j$. We assume that $N$ is ordered following $\trianglerighteq$, if not, a permutation suffices to extend the definition. This is a jungle equilibrium. Let $y$ 
    \end{proof}
\end{proposition}

Historically, the Jungle (\cite[PR]{P-R} (2007)) was introduced before the civilized jungle (\cite[RY]{RY} (2022)). We presented the latter first both because it is our main object of interested in this dissertation, and also because this very paper was written for indivisible commodities, while the former has divisible goods. Indeed, we will heavily take inspiration on the paper by Piccione and Rubinstein while allowing the civilized jungle for divisible commodities. 

\subsection{Comparison between equilibria}

Which relation do C equilibrium and jungle equilibrium have? Let us state a property of the power relation in order to answer this question. 

\begin{definition}
    A strict ordering $\trianglerighteq$ is \textbf{weakly} $\mathcal{L}$\textbf{-concave} if $\forall i,j\in N$ and $\forall\lambda\in\Lambda$:
    \[\exists i_{\lambda}\in N\setminus\{j\} \text{ s.t. } i_{\lambda}\geq_{\lambda}j \text{ and } i\triangleright i_{\lambda} \Rightarrow i\triangleright j\]
\end{definition}

Weak convexity is an extension of the basic notion of convexity. It was proposed by Richter and Rubinstein in \cite[RR]{Convex_Pref}, and it allows the definition of the property for any space without the need of an algebraic structure. In our setting it is quite intuitive if seen as: if \textit{for all} the criteria that we are using in our language if I ($i$) can find some one ($i_{\lambda}$) weaker than me and best suited than you ($j$) then I am stronger than you. Under this condition the jungle equilibrium is a C equilibrium.

% Before proving this statement let us recall some properties of $\mathcal{L}$-concavity.\footnote{Maybe we can prove it through $\mathcal{L}$-conv iff conv under continuity}

% \begin{proposition}
%     $J_{\mathcal{L}}(I)\subseteq I$
% \end{proposition}

\begin{proposition}\label{Prop: undivisible, weakly concave L}
    Let $\langle N,X,(\succeq^i)_{i\in N}, \trianglerighteq, \mathcal{L}\rangle$ be a civilized jungle with a weakly $\mathcal{L}$-concave power relation, then the jungle equilibrium is a civilized equilibrium.

    \begin{proof}
        Let $\textbf{x}$ be the jungle equilibrium. Of course in a serial dictatorship $E(\textbf{x},i)\subseteq\{j\,|\,i\triangleright j\}$, then $i$ is the strongest among $J_{\mathcal{L}}(E(\textbf{x},i))$ if $i$ belongs to it. In fact, recall that justified agents in a group are part of that group \cref{eqn: J_L}. If by contradiction $i\notin J_{\mathcal{L}}(E(\textbf{x},i))$, then $\forall\lambda\in\Lambda$ there exists $i_{\lambda}\neq i$ such that $i_{\lambda}\geq_{\lambda}i$ but also $i\triangleright i_{\lambda}$, and so by weak concavity $i\triangleright i$.   
    \end{proof}
\end{proposition}

% \color{green}{This is what R-Y\cite{RY} write. Actually, I think that the concavity of the power relation is not necessary. We gave the definition of the justified of a group as a subset of that group, it seem natural in this setting, why does someone who is not interested in the good justify himself. Therefore given our definition we have:

% \[J_{\mathcal{L}}(E(\textbf{x},i)\subseteq E(\textbf{x},i)\,\forall i\in N\]

% Then if some}\color{black}

Furthermore, if the notion of concavity is empowered the jungle equilibrium becomes the only C equilibrium.

\begin{definition}
    A power relation is \textbf{strongly $\mathcal{L}$-concave} if $\forall i,j\in N$:
    \[\exists i_{\lambda}\in N\setminus\{j\}\text{ s.t. }i_{\lambda}\geq_{\lambda}j\text{ and }i\trianglerighteq i_{\lambda}\Rightarrow i\triangleright j\]
\end{definition}

While weak $\mathcal{L}$-concavity asks the agents to be strongly separated by the power relation ($i\triangleright i_{\lambda}$), strong $\mathcal{L}$-concavity delivers $i\triangleright j$ even if $i=_{\lambda} i_{\lambda}$.

\begin{proposition}\label{Prop: undivisible, strictly concave L}
    Let $\langle N,X,(\succeq^i)_{i\in N}, \trianglerighteq, \mathcal{L}\rangle$ be a civilized jungle of strict orderings criteria with a strongly $\mathcal{L}$-concave power relation, then the jungle equilibrium is the unique civilized equilibrium.

    \begin{proof}
        Let $\textbf{y}$ be another C equilibrium. Then not being the serial dictatorship implies the existence of $i,j\in N$ such that $i\triangleright j$ and $x^j\succeq^ix^i$. Because $\textbf{y}$ is a C equilibrium $i\notin J_{\mathcal{L}}(E(\textbf{x},j))$, then $\forall\lambda\in\Lambda\,\exists i_{\lambda}\in N\setminus\{i\}$ such that $i_{\lambda}\geq_{\lambda}i$. Since $\textbf{y}$ is a C equilibrium $j\trianglerighteq j_{\lambda}\,\forall\lambda$. Then by strong concavity $j\triangleright i$.  
    \end{proof}
\end{proposition}

\newpage

% \section{Welfare theorems}

% Considered this simple economic model we want to establish whether or not the welfare theorems hold. Let us recall them heuristically: the first one states that under pretty weak hypotheses any competitive equilibrium is Pareto optimal while the second one states that under more stringent assumptions any Pareto optimal allocation can be attained as a competitive equilibrium through a suitable price vector (and share allocation). Let's talk math. We will present the welfare theorems in (almost) their most general formulation, following D. Acemoglu \cite{Growth_theory-Acemoglu}. At first we need to define the building block of this theory, i.e. a stylized economy. We set $N\in\mathbb{N}^*=\mathbb{N}\cup\{+\infty\}$ as the number of agents, $K\in\mathbb{N}^*$ as the number of commodities and $\mathcal{F}$ as the set of firms. Commodities can be taken in a continuum, but this setting is sufficient for our purposes. Each agent $i\leq N$ has a \textit{preference relation} $\succeq^i$ on the set of bundles $\mathbb{R}_+^K$ and a \textit{consumption set} $X^i\subseteq\mathbb{R}_+^K$. We assume the preference relation to be strongly monotone and continuous (WHY??) and the consumption set to be compact, convex and that it satisfies free disposal (???). We won't allow consumption to negative, the extension is straightforward. We define $\textbf{X}=\prod_{i=1}^NX^i$ the aggregate consumption set as the cartesian product of all the consumption sets. We then denote $Y^f\subseteq\mathbb{R}^K$ (???) for each $f\in\mathcal{F}$ as the production set of firm $f$. We assume each production set to be a cone and denote $\textbf{Y}=\prod_{f\in\mathcal{F}}Y^f$. At last we define the \textit{profit share} deriving from the aggregate production as $\theta=(\theta^f)_{f\in\mathcal{F}}$ where $\theta^f\in\mathbb{R}_+^N$ represents the redistribution of firm $f$'s production to each agent. We normalize to $1$ for each firm, that is $\sum_{i=1}^N\theta_f^i=1\,\forall f\in\mathcal{F}$.

% \begin{definition}
%     The tuple $\langle N, \mathcal{F}, \textbf{X}, \textbf{Y}, {\omega}, \theta\rangle$ \dots
% \end{definition}


% \newpage

\section{Civilized Jungle with divisible commodities}
% 
We now extend the setting of a civilized jungle and the relative equilibrium concept for divisible commodities. Relying on the paper by Piccione and Rubinstein \cite[PR]{P-R} we define an economy with $K$ commodities with an aggregate bundle $w=(w_1,\dots,w_K)$, where $w_k\in\mathbb{R}_+\,\forall k\in K$. Each agent $i\in N$ has a consumption set $X^i\subseteq\mathbb{R}_+^K$ and $X=(X^1,\dots,X^N)$. Therefore, a \textbf{civilized jungle} is a tuple:

\[\langle N,K,(X^i)_{i\in N}, w, (\succeq^i)_{i\in N}, \trianglerighteq, \mathcal{L}\rangle.\]

Because agents can consume more than just one commodity in different quantities, economic scenarios cannot be represented by an assignment, instead they are modeled through allocations. An allocation is a vector $\textbf{z}\in \mathbb{R}_+^K\times X$, where the first coordinate stands for the unused goods while the following for each good. Therefore, $\sum_{i=0}^Nz_i=w$. We now want to adapt the civilized equilibrium concept by redefining the notion of "envy". In this context, agent are not envious, because their economic situation can be, and realistically is always, composed of divisible commodities. Given an allocation, an agent can "dream" about another agent if she sees in the dreamed one's property a more preferred allocation. We enrich the envious relationship as agents have more complex behaviors in this setting. For instance, a monkey can dream about another primate not because she has a certain variety of bananas, but because she has a certain quantity of a certain type of bananas. This situation requires a broader concept of envy, as the difference in goods is not its only driver. To keep it simpler, we do not allow coercion on multiple agents. I define formally the concept of a dreamer.

\begin{definition}
    Given an allocation $\textbf{z}$ and $i,j\in N$ we say that $i$ \textbf{dreams} about $j$ if:

\[\exists y^i\in X^i \text{ such that } y^i\succeq^iz^i \text{ and } y^i\leq z^i+z^j\]
\end{definition}

If commodities are indivisible agents' preferences must rank them in function of their unitary value; there is no difference between 2 bananas and 1 stick, you either prefer the banana or the stick. If commodities are divisible quantities play a fundamental role, it may be the case that you prefer one stick more than one banana and prefer two bananas over one stick. This wider range of possibilities suggests the necessity of enlarging the notion of envy. An agent can build a more preferred allocation by taking a part of another agent's property instead of switching the whole allocation. Since we allow for this instance to take place we named it as \textit{dreaming} about another agent.

We define the new set of those who dream about someone in a given allocation:

\[ D(\textbf{z},i) = \left\{ j\in N\,|\,\exists y^i\in X^i \text{ s.t. } y^i\succeq^iz^i,\, y^i\leq z^i+z^j\right\}.\]

We can now define the jungle equilibrium concept, initially defined by \cite[PR]{P-R}.

\begin{definition}\label{def1}
    An allocation $\textbf{z}$ is a \textbf{jungle equilibrium} if $\nexists i,j\in N$ s.t. $i\triangleright j$ and $\exists y^i\in X^i$ s.t. $y^i\succeq^iz^i$ and:
    \[y^i\leq z^i+z^j\text{ or }y^i\leq z^i+z^0\]
\end{definition}

We can reformulate the definition \ref{def1} as follows, in terms of dreamers.

\begin{definition}\label{def2}
    An allocation $\textbf{z}$ is a \textbf{jungle equilibrium} if:
    \begin{itemize}
        \item $\forall i\in N\,:i\triangleright j\,\forall j\in D(\textbf{z},i)$
        \item $D(\textbf{z},0)=\emptyset$
    \end{itemize}
    
\end{definition}

In a jungle equilibrium, every agent has to be the strongest among those who dream of her. As before, civilization imposes a friction between dreaming about someone and coercively taking some of her properties. In a \textit{civilized} jungle, agents have to justify themselves through at least one criterion to "fight" for commodities. Once agents are justified, jungle law prevails.

\begin{definition}\label{Cequilibrium}
    An allocation $\textbf{z}$ is a \textbf{civilized equilibrium} if $\forall i\in N$ the following hold:
    \begin{enumerate}
        \item $i\in J_{\mathcal{L}}(D(\textbf{z},i))$
        \item $i\triangleright j\,\forall j\in J_{\mathcal{L}}(D(\textbf{z},i))$
        \item $D(\textbf{z},0)=\emptyset$
    \end{enumerate}
\end{definition}

The first two conditions were previously explained. The third one assures that the unused goods aren't useful for anyone, as if they were, monkeys would take them. We now present some examples that will help to give a better understanding of this setting.

%It can be restated as:

%\[\forall i\in N\,:\, \exists y^i\in X^i \text{ s.t. } y^i\succeq^iz^i,\, y^i\leq z^i+z^0 \,\Rightarrow\,\forall\lambda\in\Lambda\exists j_{\lambda}\in D(z,0) \text{ s.t. } j_{\lambda}\triangleright i\]

%I point out this less concise definition in order to highlight that agents can dream about exploiting unused goods in equilibrium, but they are not justified in doing so.


Before doing so we point out once again the further step required by civilization: in equilibrium, agents have to be the strongest among the justified dreamers, not just among those who dream. We recall that agents justified among a group are defined to be part of that group. The definition could be also given without this condition, letting agents to be justified in groups of which they are not part, but it seems unreasonable in our setting, given that only dreamers are interested in being justified.

\begin{example}[Uncivilized languages]
    Not all languages provide actual civilization. Let us consider a singleton language $\mathcal{L}=\{\lambda\}$, how is it going to affect the economy? Let us consider an allocation $z$ and look at the C equilibrium definition \ref{Cequilibrium}. For every agent $i$, the set of justified dreamers has to be a singleton ${\mathcal{L}}(D(z,i)=\{j\}$ where $j>k$ for every $k\in D(z,i)$. If we want $z$ to be a C equilibrium then $i\triangleright j$. Therefore the unique C equilibrium in this economy is the serial dictatorship according to the language. We call such a language an \textit{uncivilized} language, since it just changes the power relation, without introducing any friction between dreams and power. This result perfectly mimics \cref{prop: L singleton} What if the language takes the opposite form? Let us consider a language $\mathcal{L}=\mathcal{S}_N$, that is the set of all permutation over agents.  Under this language, every agent is going to be justified in every subset since there will always be a criterion with respect to which she is the strongest. Because of this, the language does not change anything on the equilibrium side, its presence is neutral and agents are going to construct equilibria thinking about being justified and the economy falls back into a standard Jungle.

    These two examples show that real civilization has to come at a price, they have to be exclusionary to provide useful justifying criteria.
\end{example}

Rubinstein and Yildiz define the "I am how I am" criterion as dichotomous language $\mathcal{L}=\{m_i:\,i=1,\dots,N\}$ where $\{m_i\}$ defines agent $i$ as strictly stronger than the others, who are equally strong. This language provides the same result as $\mathcal{S}_N$. Indeed, what implies neutrality of a language is the justification of every agent in every group, which is attained by the "I am who I am" language.

\begin{example}[Strictly monotonic preferences]
    If we deal with strictly monotone preferences there is a unique C equilibrium. It will be the allocation that gives the whole endowment to the strongest ($\triangleright$) among those justified by the language within the set of all agents. That is, $\mathcal{L}=\{\lambda_{\lambda},\,\lambda\in\Lambda\}$ selects $m\leq n$ agents who are the strongest for a criterium, formally $J_{\mathcal{L}}(N)=\{j:\exists \lambda\in\Lambda\,j>^{\lambda}k\,\forall k\in N\}$. Among those, the strongest with respect to the power relation $\triangleright$ is the one who'll get $w$. That will be the only C equilibrium. Indeed, if $z$ is a C equilibrium then every agent $i$ who has got positive consumption will have everyone dreaming about him, therefore
    
    \[J_{\mathcal{L}}(D(z,i))=J_{\mathcal{L}}(N)=\{j:\exists \lambda\in\Lambda\,j>^{\lambda}k\,\forall k\in N\},\]

    but $i\triangleright j$ for every $j\in J_{\mathcal{L}}(N)$, which implies that the only agent who can have positive consumption is the strongest among $J_{\mathcal{L}}(N)$. Since he is the only one, he is going to consume the whole aggregate endowment. Clearly, if her consumption set does not allow for the whole endowment to be consumed, then what is left goes to the second strongest by the same reasoning. We apply iteratively the reasoning until everything is consumed or every has fully satisfied her consumption set.    
\end{example}

The last example we want to show is proposed by Rubinstein and Yildiz in \cite[RY]{RY}, and it combines multiple dichotomous languages in a nested organization.

\begin{example}[Nested dichotomous languages]
    Let us imagine a civilized equilibrium with a dichotomous language such that the set of properties are nested. That is, we imagine an ordering of the agents $i_1,\dots,i_N$ such that $\phi^{i_N}\subset\dots\subset\phi^{i_1}$. This civilized jungle has a unique equilibrium, the serial dictatorship according to $i_1,\dots,i_N$. Again, essentially we can apply the same reasoning used in the proof of Proposition \ref{prop: strict ordering} to show that it is indeed an equilibrium and is unique. 
\end{example}

\subsection{Comparison between equilibria}

What is the relation between a civilized and an uncivilized equilibrium? Does their relation for indivisible commodities still hold? Yes it does, but we lose uniqueness.

\begin{proposition}\label{Prop: jungle eq is C in weakly concave}
In a weakly concave jungle, a jungle equilibrium is civilized.

    \begin{proof}
        The second condition is fulfilled. $J_{\mathcal{L}}(D(\textbf{z},\cdot))\subseteq D(\textbf{z},\cdot)$, therefore being the strongest in the latter implies being stronger in the justified envious, if part of it. Then by contradiction as before we prove the belonging.
    \end{proof}
\end{proposition}

Of course, in this setting uniqueness under strict concavity does not hold. The proof of \cref{Prop: undivisible, strictly concave L} by R-Y\cite{RY} breaks instantly when the existence of another C equilibrium implies that there is at least one powerful envious, meaning that she is stronger than the envied one. It is not true when goods are more than just units, possibly there are multiple best bundles for each agent.

\newpage

\section{Civilized Welfare Theorems}

I now investigate whether or not the welfare theorems hold in a Civilized Jungle. As a first step, we have to adapt the statement of the theorems to this setting. Evidently, a competitive allocation is a civilized equilibrium while the concept of \textit{Pareto efficiency} is intended as follows.

\begin{definition}
    An allocation $\textbf{z}$ is \textbf{Pareto efficient} if does not exist another $\textbf{y}$ allocation s.t:
    \[y^i\succeq^iz^i\,\forall i\in N \text{ and } \exists j\in N \,: \,y^j\succ^jz^j \]
\end{definition}

That is, an allocation is Pareto efficient if no one can improve her situation without making anyone else worse off. 

In further sections, production will be introduced in the jungle setting, until then the welfare theorems are stated in their simplified versions.

\subsection{First civilized theorem}

The first theorem can be interpreted as:

\begin{center}
    A civilized equilibrium is Pareto efficient.
\end{center}

We know from \cite*[PR]{P-R} that unique jungle equilibria are indeed Pareto efficient, from now on just efficient. Rubenstein and Yildiz \cite[RY]{RY} show that a strong result holds for civilized jungles.

\begin{proposition}\label{Prop: no PE}
    Given a civilized jungle with a language of strict orderings such that there exists no agent $i,j$ where one is ranked right above the other and the opposite relation holds for the language. If the power relation is not weakly concave then there exists a preference profile such that there exists no pareto efficient C equilibrium. 
\end{proposition}

The proof is rather long and not of much interest for our purposes. As the authors point out, Proposition \ref{Prop: no PE} together with Proposition \ref{Prop: undivisible, weakly concave L} implies that if a civilized jungle has a language of strict orderings then weak $\mathcal{L}-$concavity of the power relation is almost a necessary and sufficient condition for the existence of a Pareto efficient civilized equilibrium for every preference profile.  
Since we already lose the first welfare theorem power for indivisible commodities we shift our attention to the second one, which has a civilized version in \cite[PR]{P-R}.

\subsection{Second civilized theorem}

The second welfare theorem, in its standard formulation, guarantees for every Pareto efficient allocation, under suitable assumption, the existence of a price vector and an endowment that sustains that allocation as a competitive one. In a jungle, even if civilized, the only currency is brute force, then prices and personal endowments are substituted by a power relation. Therefore we can restate the theorem as:

\begin{center}
    For every Pareto efficient allocation there exists a power relation which sustains the allocation as a civilized equilibrium. 
\end{center}

If dealing with non-divisible goods, \cite[RY]{RY} have proved an analogous version of the second welfare theorem. The two key hypotheses are strict orderings as languages and restraining efficient allocation to \textit{J-costrained} efficient allocation. The first one guarantees a clear power relation. The latter is an important and necessary constraint: in a civilized equilibrium each agent has to justify herself among those who envy her, otherwise she cannot use her force against them. This assumption is not necessary for an efficient assignment. It is therefore necessary to introduce the following class of assignments.

\begin{definition}
    An allocation $\textbf{z}$ is J-constrained if $i\in J_{\mathcal{L}}(E(\textbf{z},i))\,\forall i \in N$.
\end{definition}

Once we constrain an efficient assignment to a language we can inquire under which conditions on $\mathcal{L}$ each efficient assignment is an equilibrium. Turns out that for divisible commodities the adapted proof of Proposition \ref{Prop: jungle eq is C in weakly concave} from \cite[RY]{RY} breaks immediately. The idea is to build the power relation as a completion of a non-cyclic binary relation over a subset of $N$. In particular, the subset over which the (possibly incomplete) binary relation is defined is such that it guarantees the assignment to be C equilibrium. I now present the theorem from \cite[RY]{RY} and then show why it does not hold in our more general setting.

% This idea does not hold for divisible commodities because we are not able to construct 

% This binary relation mimics the necessary power relations between agents to make the efficient allocation an equilibrium. 

\begin{theorem}
    Let $\langle N,(X^i)_{i\in N}, (\succeq^i)_{i\in N}, \mathcal{L}\rangle$ be a tuple as above. Then, for every J-constrained efficient assignment \textbf{x} there exists a power relation $\trianglerighteq$ such that \textbf{x} is a C equilibrium for the corresponding civilized jungle. 

    \begin{proof}
        Let \textbf{x} be a J-constrained efficient assignment. Let $P$ be a binary relation over $N$ such that for each $i,j\in N$ then $jPi$ if $i$ envies $j$ and she is justifiable in $E(\textbf{x},j)$. If we show $P$ to be non-cyclic, then it is a pre-order over $N$. This comes from a standard result in order theory\footnote{Varian (1974) for non cyclic implies completable}, I will talk more about it later in this section. By completing $P$, we'd get a power relation $\trianglerighteq$, which sustains $\textbf{x}$ as a C equilibrium. Indeed:
        \begin{itemize}
            \item $i\in J_{\mathcal{L}}(E(\textbf{z},i))$ for all $i\in N$ because $\textbf{x}$ i J-constrained
            \item If $i$ is justifiable in $E(\textbf{z},j)$, then $jPi$, then $j\triangleright i$
            % \item Clearly $D(\textbf{z},0)=\emptyset$, because $\textbf{z}$ is efficient.
        \end{itemize}

        Let us show that $P$ is non-cyclic. Suppose by contradiction that for some $I=\{1,2,\dots,m\}$ we have $1P2P\dots PmP1$. Let us define the allocation \textbf{y} as $y^i=x^{i-1}$ for $i\in I$ and $y^i=x^i$ for $i\notin I$. The assignment \textbf{y} is justifiable and pareto dominates \textbf{x}. The latter is obvious by construction. By recalling that the operator $L_{\mathcal{L}}$ is monotone decreasign\footnote{Meaning that $A\subseteq B\Rightarrow J_{\mathcal{L}}(A)\supseteq J_{\mathcal{L}}(B)$} we prove \textbf{y} to be justified, indeed for $i\in N$ then:
        \begin{itemize}
            \item If $i\in I$ then $E(\textbf{y},i)\subseteq E(\textbf{x},i-1)$ 
            \item If $i\notin I$ then $E(\textbf{y},i)\subseteq E(\textbf{x},i)$
        \end{itemize}

        But $i$ is justified in both $E(\textbf{x},i)$ and $E(\textbf{x},i-1)$, because \textbf{x} is justified and by definition of $P$.
    \end{proof}
\end{theorem}

The whole proof relies on the possibility of constructing an assignment \textbf{y} that pareto dominates \textbf{x}. For divisible commodities, two main problems arise. We first think about non-civilized jungles. We call the first one \textit{reciprocal dreaming} the instance in which two agents dream each other. This situation could clearly occur even in the non-divisible goods setting, but only in non-efficient assignments. Indeed, if two monkeys envy each other they can simply switch their bananas and get a more efficient assignment. When bananas are divisible, reciprocal dreaming does not imply inefficiency. But if in an efficient allocation two agents are reciprocal dreamers, then no power relation will sustain the allocation as an equilibrium. Indeed, every power relation would not be strict as for the reciprocal dreamers $i,j$ the equilibrium condition would imply $i\triangleright j$ and $j\triangleright i$. We found a necessary condition. To formally express it we define reciprocal dreaming.

\begin{definition}\label{Def: reciprocal dreaming}
    Let $z$ be an allocation. Two agents $i,j\in N$ are \textbf{reciprocal dreamers} if:

    \[i \in D(z,j) \text{ and } j \in D(z,i).\]

    An allocation is \textbf{non-reciprocal} if no reciprocal dreamers exist. 
\end{definition}

Non-reciprocality is a key concept in C-equilibria since it drastically shapes the power relation of the Jungle. 


\begin{proposition}\label{Jungle implies no reciprocality}
    Let $z$ be a Jungle equilibrium, then $z$ is non-reciprocal.

    \begin{proof}
        By absurd. Let $i,j\in N$ be reciprocal dreamers. By definition of Jungle equilibrium

        \[j\in D(z,i)\Rightarrow i\triangleright j\]

        and 

        \[i \in D(z,j) \Rightarrow j\triangleright i.\]

        Then $\triangleright$ is not antisymmetric $\lightning$. 
    \end{proof}
\end{proposition}

If we impose non-reciprocal dreaming can we prove the theorem? Not yet, the very essence of the more general setting prevents the allocation \textbf{y} from being constructed. While for non-divisible commodities monkeys could just switch different bananas, in this setting what is dreamed by two monkeys could be unfeasible together. We therefore should follow a different path, a straightforward extension of the proof is impossible.

Actually, a general result holds: the II welfare theorem does not hold for uncivilized jungles with divisible commodities. Given an allocation, we can sustain it as a C equilibrium only if in the subset of dreamers there are no reciprocal dreamers (we can impose this condition) and those who dream are weaker than the dreamed one. We can easily construct a Pareto efficient allocation where the two previous conditions are fulfilled, but it is not an equilibrium.
% , but the binary relation resulting doesn't satisfy OWC, which will be studied later, and therefore cannot be extended to a complete order (because of \ref{Sziplrajn}).

\begin{example}
    Let three monkeys fight for one banana, one nut, and one stick. Monkey $A$ prefers one banana and the stick over everything, monkey $B$ prefers the banana and the nut over everything else, and monkey $C$ prefers the nut and the stick. The preference relations are:

    \[\begin{cases}
        (1,0,1)\succ (1,0,0) \text{ strictly better than all others, over which she is indifferent.} \\
        %\succ^a(0,0)\sim^a(0,1)\sim^a(1,1) \\
        (1,1,0)\succ (0,1,0)  \text{ strictly better than all others, over which she is indifferent.} \\
        %\succ^b(0,0)\sim^b(0,1)\sim^b(1,0) \\
        (0,1,1)\succ (0,0,1)  \text{ strictly better than all others, over which she is indifferent.}
        %\succ^c(0,0)\sim^c(1,0)\sim^c(1,1)
    \end{cases}\]

    Consider the allocation $z=((1,0,0),(0,1,0),(0,0,1))$. It is Pareto efficient (the only allocations where someone is strictly better off are obtained if $B$ takes the banana from $B$, or if $C$ takes the nut from $B$ or $A$ takes the stick, in all instances, the robbed are worse off). There is no reciprocal dreaming, $B$ dreams about $A$ and is dreamed by $C$, while $A$ only dreams about $C$. To have a power relation $\triangleright$ sustaining $z$ as a jungle equilibrium it has to be the case that

    \[A\triangleright B\text{ and } B\triangleright C.\]

    But then it must be the case that $A\triangleright C$, which is absurd because $A$ dreams about $C$'s stick.
\end{example}

In the next chapter, we study how to break these circles that prevent an allocation from being sustained as an equilibrium. This will be done by civilization, which will be part of the legislative power of jungle institutions.


% WE CAN TRY TO DEAL WITH OWC BREAKING WITH A LANGUAGE. We can impose a language such that whenever OWC is broken by transitivity the more powerful dreamer (impossible in equilibrium) is excluded from the fight by civilization.

% This approach doeesn't work...


% If the language can be implemented by external forces, as a jungle policy, then the II welfare theorem holds.


% A stronger, or useless, result holds.

% \begin{proposition}
%     For every allocation $z$ and every power relation, there exists a language that supports $z$ as a civilized equilibrium

%     \begin{proof}
%         Put just one order? Think about it, should be straightforward.

%         It is sufficient to construct a language that prevents anyone from acting coercively over dreamed agents.  Does such a language exist? 
%     \end{proof}
% \end{proposition}

% We can ask the language to satisfy a property that ensures OWC. We want $\mathcal{L}$ s.t. $\forall \{a,b\}\in \mathcal{R}_z$ then:

% \[\forall\lambda\in\Lambda\,\exists c_{\lambda}\in D(z,a)\,:\,c_{\lambda}>_{\lambda}b\]

% Where:

% \[\mathcal{R}_z=\left\{\{a,b\}\,|\,(a,b),(b,a)\in \hat{\mathcal{P}}_z\right\}\]

% And:

% \[\hat{\mathcal{P}}_z=\left\{(a,b)\,|\, aT(\trianglerighteq)b\right\}\]

% Where $\triangleright$ is defined by:

% \[a\triangleright b\text{ if }b\in D(z,a)\]

% Under this condition, which forces OWC, we can prove the II welfare theorem.


\newpage


\section{Jungle policies}

In this section, we study how jungle institutions can implement civilized equilibria. We think of a jungle institution as a policymaker who can apply power relations and languages. Such an entity will be able, under certain circumstances, to impose an allocation as an equilibrium. The main concept is \textbf{civilized compatibility} (\textit{C-compatibility}) of a language with respect to an allocation. We will prove that every allocation that admits a C-compatible language can, almost, be a C equilibrium under that language.


Let us take a little detour on the basics of order theory, which will help us precisely spot where we can ask for sufficient conditions to be matched for a completion to a preorder. This procedure will be the main idea behind the implementation of a language and a power relation to get a C equilibrium.

% We'll make use of the following definition.

% \begin{definition}
% Let $\succeq$ be a binary relation on a non-empty set $X$, $\succ$ be the asymmetric part of $\succeq$ and $T(\succeq)$. We say that $\succeq$ satisfies \textbf{only weak cycles} (OWC) if:
% \[xT(\succeq) y\Rightarrow \lnot(y\succ x)\]
% \end{definition}

% As its name suggests, the OWC condition is a weaker condition than acyclicity. Recall that for any binary relation $R$ its transitive closure is denoted by $T(R)$. The transitive closure of $R$ is the smallest binary relation that contains $R$ (in the sense that $xRy$ implies $xT(R)y$ for all $x,y$) is transitive and any transitive binary relation containing $R$ also contains $T(R)$. 

We'll need the following modification of the classical extensions theorem from Sziplrajn.

\begin{theorem}\label{Sziplrajn}
    For a nonempty set $N$ and an irreflexive, transitive, and antisymmetric binary relation $\succ$ there exists a complete extension of $\succ$ that preserves the properties.
\end{theorem}

The details of this result and its proof are discussed in the appendix.

% \begin{proposition}\label{Sziplrajn}
% Let R be a binary relation on a non-empty set X. Then R can be extended to a complete preorder if and only if it satisfies OWC.

% \begin{proof}
%     I will write it. Via Sziplrajn extension theorem and a good reference I found.\cite{MarkDean}
% \end{proof}
% \end{proposition}


% We adjust the path we are following. As seen, the II fundamental theorem is substantially false for divisible commodities. Anyhow we can restrict ourselves to smaller allocations that can be sustained as C equilibria. that admit a \textit{C-compatible language}.


Having acquired the necessary mathematical tools, we now turn our attention to C equilibria and how they can be sustained by power relations. Let us consider a generic allocation $z$ and study when it can be sustained a s C equilibrium. We already noticed in Proposition \ref{Jungle implies no reciprocality} that a Jungle equilibrium implies no reciprocal dreaming in the economy. Since we are now working with civilized equilibria we have to adapt this result. We will use the term equilibrium while referring to C equilibrium from now on. Let us recall that for each pair of agent $i,j\in N$ in a C equilibrium it must hold that if they are reciprocal dreamers then at least one of them is not justified in the set of dreamers of the other. Indeed, to maintain antisymmetry of the power relation if $i\in D(z,j)$ and $j\in D(z,i)$, but $z$ is an equilibrium then

\[j\triangleright k\,\forall k\in J_{\mathcal{L}}(D(z,j)),\]

and 

\[i\triangleright k\,\forall k\in J_{\mathcal{L}}(D(z,i)).\]

If both $i$ and $j$ are justified in $D(z,j)$ and $D(z,i)$ respectively, then

\[j\triangleright i \text{ and } i\triangleright j,\]

which contradicts antisymmetry. It is now clear that whenever reciprocal dreaming occurs in an allocation we entrust the language to break it, to sustain it as an equilibrium. It is crucial to note that, given a generic allocation if we aim to impose it as an equilibrium every power relation $\triangleright$ must satisfy

\begin{equation}\label{Cond C eq}
    i\triangleright j\,\forall j\in J_{\mathcal{L}}D(z,i)\setminus\{i\}.
\end{equation}

Therefore, we must address all the instances in which this condition contradicts the very nature of $\triangleright$. A necessary condition on each couple $(\triangleright,\mathcal{L})$ of power relation and language is \ref{Cond C eq}. We take such a tuple. By definition of $\triangleright$ we now have a binary relation, probably incomplete, on $N$. If we can completely extend it to $N$, showing it to be irreflexive, transitive, and antisymmetric we will have, almost, constructed a civilized power hierarchy, the tuple, that sustains $z$ as an equilibrium. Let us extend the power relation $\triangleright$, preserving the necessary properties. We now make use of Theorem \ref{Sziplrajn}. If we construct an extension of $\triangleright$ to a transitive, irreflexive, antisymmetric relation we can apply the slight modification of Sziplrajn's Theorem in \cref{Sziplrajn} and get the desired power relation. For transitivity, we take the transitive closure of $\triangleright$, denoted by $T(\triangleright)$. Is this relation irreflexive and antisymmetric? At this point, civilization itself comes into play; it will prevent $T(\triangleright)$ from not being irreflexive and antisymmetric, allowing us to invoke Sziplrajn's Theorem. Actually, we will see that preventing the lack of antisymmetry will also impose irreflexivity. Let us suppose $T(\triangleright)$ is not antisymmetric, i.e. there exists no two agent $i,j$ such that 

\[iT(\triangleright)j \text{ and }jT(\triangleright)i.\]

By definition of $\triangleright$ for every agents $k,l\in N$

\[k\triangleright l \Leftrightarrow l\in J_{\mathcal{L}}(D(z,k)),\]

therefore, if $iT(\triangleright)j$ it must be the case that $i$ and $j$ are connected by a chain of justified dreamers from $i$ to $j$, that is there exists $k_{h}\in N$ with $h=1,\dots,H$ where $H<N-1$ such that $j$ dreams about the last $k$, each $k$ dreams about the previous one, and the first $k$ dreams about $i$, all justified. 

% This is not only a sufficient cndition for $iT(\triangleright)j$ to hold but it is necessary. 

\begin{proposition}\label{Prop: cond on transitive closure}
    Let $z$ and allocation and $\triangleright$ defined as in \ref{Cond C eq}. Then $iT(\triangleright)j$ only if either $i\triangleright j$ or $\exists \{k_h\}_{h=1}^H$ for $k_h \in N$ and $H<N-1$ such that

    \[j\in J_{\mathcal{L}}(D(z,k_H)),\, k_H \in J_{\mathcal{L}}(D(z,k_{H-1})),\,\dots,\, k_1\in J_{\mathcal{L}}(D(z,i)).\]

    \begin{proof}
        Let us suppose that neither $i\triangleright j$ nor there exists a chain of justified dreamers connecting $i$ and $j$. Then we take $\triangleright'\overset{def}{=}T(\triangleright)\setminus\{i,j\}$. This is an extension of $\triangleright$ since $i\triangleright j$ does not hold and all other relations are preserved from the transitive closure. If we prove it to be transitive we get the absurd. The only lack of transitivity can occur for $i$ and $j$, since for every other couple of agents it is guaranteed by $T(\triangleright)$. If, by absurd, there exists a chain of $k_h$ such that

        \[i\triangleright'k_1, k_1\triangleright'k_2\,\dots,\,k_H\triangleright'j,\]

        then either all relations are already in $\triangleright$, and therefore it contradicts the initial absurd hypothesis, or there exists a couple of the chain for which the relation only holds in $\triangleright'$. If that is the case we apply this reasoning again. Since there are a finite number of agents we end up back with $i\triangleright'j$, absurd.
    \end{proof}
\end{proposition}

We can now break via civilization the lack of antisymmetry; we can ask just for one couple in the dreaming loop created by reciprocal dreamers to be broken by civilization. We call this C-compatibility between and allocation and a language.

\begin{definition}
    Let $z$ and allocation and $\mathcal{L}$ a language. They are \textbf{C-compatible} if for all couples of reciprocal dreamers $i,j\in N$ one of them is not justified in the set of dreamers of the other. 
\end{definition}

A more formal, less intuitive, but more handy definition is the following.

\begin{definition}
    Let $z$ and allocation and $\mathcal{L}$ a language. They are \textbf{C-compatible} if for all couples of reciprocal dreamers $i,j\in N$ then one of the following holds

    \begin{enumerate}
        \item $\forall \lambda\in \Lambda$ there exists $k_{\lambda}\in N\setminus\{i,j\}$ such that $k_{\lambda}>^{\lambda}j$,
        \item $\forall \lambda\in \Lambda$ there exists $l_{\lambda}\in N\setminus\{i,j\}$ such that $l_{\lambda}>^{\lambda}i$.
    \end{enumerate}
\end{definition}

C-compatibility breaks those chains that prevented the transitive closure from being antisymmetric. Furthermore, it also prevents a lack of irreflexivity. Indeed, in light of Proposition \ref{Prop: cond on transitive closure}, if $iT(\triangleright)i$ it must be the case that either $i\triangleright i$, which is impossible by definition, or there exists a chain of justified dreamers from $i$ into itself. For a C-compatible language, this chain is necessarily broken somewhere. 

We have almost proven a jungle policy theorem, the last ingredient needed is J-constrainment. If we impose it, together with C-compatibility, we can build a power relation that sustains the allocation as an equilibrium. We impose that no unused goods are wanted by anyone since this clearly prevents any jungle policy from sustaining the allocation as an equilibrium.



% Formally, what do we mean by C compatibility of a language? Let $z$ be an allocation and let us introduce a collection of pairs of agents:

% \[\mathcal{R}_z=\big\{\{a,b\}\,|\,aT(\trianglerighteq)b\text{ and }bT(\trianglerighteq)a\big\}\]

% Where $\triangleright$ is defined by:

% \begin{equation}\label{Def: triangle_def}
%     a\triangleright b\text{ if }b\in D(z,a)
% \end{equation}

% That is, $\mathcal{R}_z$ is the collection of all pairs of agents that are "equally powerful" under the transitive closure of $\triangleright$, where this binary relation is defined to satisfy the C equilibrium condition. If we aim at a C-equilibrium a necessary condition is that the power relation $\triangleright$ satisfies \ref{Def: triangle_def}. This requirement directly comes from the equilibrium definition \ref{Cequilibrium}. Then, If two agents $a,b$ are reciprocal dreamers the relation $\triangleright$ is such that

% \[a \triangleright b \text{ and } b \triangleright a,\]

% which contradicts asymmetry. If no pair agents reciprocally dream themself this issue does not arise and the construction of a power relation to sustain the allocation  as a C-equilibrium.

% \begin{definition}
%     Let $z$ be an allocation. Two agents $i,j\in N$ are \textbf{reciprocal dreamers} if:

%     \[i \in D(z,j) \text{ and } j \in D(z,i).\]

%     An allocation is \textbf{non-reciprocal} if no reciprocal dreamers exist. 
% \end{definition}

% Non-reciprocality is a key concept in C-equilibria since it drastically shapes the power relation of the Jungle. 
% % Furthermore, it is incompatible with $J$-constraining.

% % \begin{proposition}\label{Prop: Reciprocability and compatibility}
% %     Let $z$ be an allocation, if it is reciprocal then it is not $J$-constrained.
% % \end{proposition}

% Nevertheless, in many instances, reciprocal dreaming may happen, we, therefore, aim to solve this issue via civilization. The policy value of a language decides who is going to prevail between potentially equally strong agents. To do so, we have to break the equivalent power relation coming out from \ref{Def: triangle_def}.

% \begin{definition}\label{c-comp}
%     Let $z$ be an allocation. A language $\mathcal{L}$ is \textit{C-compatible} if $\forall \{a,b\}\in \mathcal{R}_z$ then $\forall\lambda\in\Lambda$ either

%     \[\exists c_{\lambda}\in D(z,a):\, c_{\lambda}>_{\lambda}b \text{ or } \exists d_{\lambda}\in D(z,b):\, d_{\lambda}>_{\lambda}a,\]
% \end{definition}

% Few words need to be spent on C-compatibility. It is a strong hypothesis on the relation between an allocation and a language; we require the language to break the power relation between every equally powerful agent, under the C equilibrium condition. C-compatibility makes one of the reciprocal dreamers not justifiable among those who dream about the other one, this does not mean that it is not justifiable among other sets of individuals. 

% ---Make an example.

% {\color{gray}{}
% rec+cc $=>$ J-UNconstraining. 

% worse: cc $=>$ uncon


% Then: 
% z C-eq $=>$ triangle satisfies \ref{Def: triangle_def} with $L$ $=>$ $a\triangleright b\Rightarrow b\notin J_{\mathcal{L}}(D(z,a))$ $=>$ either $b\notin D(z,a)$ or $\forall\lambda$ b is not the strongest among $D(z,a)$.  
% z C-eq $=>$ every agent strongest in $D(z,a)$ for some $\lambda$.

% ADJUST THE NOTION OF C-COMPATIBILITY. Not true but get it. CLEARLY POINT OUT THAT \textit{C-compatibility makes the one of the recdream not justifiable in theset of dreamers of the other one, this does not mean it is not justifiable in other sets. Make example.}}

% A C-compatible language is a sufficient tool to sustain an allocation as a C equilibrium. 

% Another important feature of an allocation is \textit{reciprocality}. 

%  Civilization allows us to handle reciprocality of an allocation by breaking this relation, but it is incompatible with $J$-constraining. 




% {\color{red}{FIX. DO ENTIRE PROOF. DO IT IN A PROPER MANNER. REQUIRE J CONSTRAIN. REQUIRE MARKET CLERANCE.} reciprocal dreaming IMPORTANTNTNTNTNT!!!}
% Address reciprocal dreaming. If it happens then z is not J constrained. Create Proposition about this. Theorem 5.2 holds if no RD. If it happens make another theorem which studies this instance. It should be something like: for RD language chooses one to be stronger (JUNGLE INSTITUTION CHOOSE ALSO LANGUAGE, SOMETHING LIKE LEGISLATIVE POWER OF POLICY MAKERS) and then for the other require C-compatibility. Finally, create a corollary (or final theorem) which states a general form ( which is just this second theorem...?). 


\begin{theorem}
    Let $\langle N,K,(X^i)_{i\in N}, w, (\succeq^i)_{i\in N} \rangle$ be a tuple as above. Then, for every $(\mathcal{L},z)$ where $z$ is $J$-constrained allocation with no unused-wanted goods, and $\mathcal{L}$ is a C-compatible language, then there exists a power relation $\triangleright$ such that \textbf{z} is a C equilibrium for the corresponding civilized jungle.
    \begin{proof}
        Let us consider the civilized jungle given by the tuple and the C-compatible language. Since $z$ is J-constrained we can define $\triangleright'$ over a subset of $N$ pairs as follows:

        \[i\triangleright' j \text{ iff } j\in J_{\mathcal{L}}(D(z,i)).\]

        Since $\mathcal{L}$ is C compatible, $T(\triangleright')$ is irreflexive and antisymmetric. By Theorem \ref{Sziplrajn} there exists a complete extension of $T(\triangleright')$, which we call $\triangleright$. The allocation $z$ is a C equilibrium in the civilized Jungle defined by $\triangleright$ and $\mathcal{L}$ since conditions 1 and 3 of Definition \ref{Cequilibrium} are imposed by hypothesis, while condition 2 is imposed by the definition of $\triangleright'$.
        
        % If we prove $\triangleright$ to satisfy only weak circles we can extend it to a complete preorder. Let us suppose by absurd that:
        
        % \[iT(\trianglerighteq)j\text{ and } j\triangleright i\]

        % By definition of the binary relation $i\in J_{\mathcal{L}}(D(z,j))$. Furthermore $jT(\trianglerighteq)i$. Therefore $\{i,j\}\in\mathcal{R}_z$, which implies that $i$ is not justifiable in $D(z,j)$. 
    \end{proof}
\end{theorem}

Let us note that the fact that no wanted commodity is unused is, for example, implied by Pareto efficiency. Nevertheless, how Pareto efficiency behaves in terms of the existence of a C-compatible language is yet to be studied. 

% In the proof we implicitly use the fact that the binary relation defined is the restriction to the justified dreamers of the one defined in definition \ref{c-comp}. That is why the last implication holds. 

\vspace{3mm}

The interpretation of this theorem is straightforward. We suppose that a jungle institution can distribute power and impose a language. If a jungle institution wants to attain a specific allocation as a civilized equilibrium it has just to check whether the allocation admits a C-compatible language. If so, the C-compatible language and the power relation derived from the C equilibrium definition itself will define the civilized jungle under which the allocation is a civilized equilibrium. 

\vspace{3mm}

As a consequence, it would be interesting to study sufficient conditions on the existence of a C-compatible language for a given allocation. We could also investigate if the existence of a C-compatible language to an allocation implies some necessary conditions on the allocation itself. Those conditions would, ideally, help jungle institutions understand whether a certain allocation is actually supportable as an equilibrium or not.


% SHOULD ADD TO DEFINITION OF C EQUILIBRIUM $J_{\mathcal{L}}(D(z,i)\cup\{i\})$

\newpage

% \newpage

% \section{Working monkeys}

% I now implement \textit{production} in a civilized jungle with divisible commodities, from now on just a civilized jungle.

% P-R propose that the endowment $\omega$ be replaced by a set $Y$ which is a collection of endowments. $Y$ stands for the production component of the jungle meaning that it shapes the agents' behavior by defining the goods' quantity. In this representation of the production side of the jungle, the endowments are defined a priori, as the power relation. My goal is to describe an endogenous production in a civilized jungle.   

% \dots

% \newpage

\newpage

\section{Conclusive Remarks}

In April 2024, Ariel Rubinstein published a book called "\dots" together with Michael Richter. In it they present four models of non-canonical economies, one of them is the Jungle. 

\newpage

\nocite{*}
\printbibliography[]
\addcontentsline{toc}{section}{References}

\newpage

\section*{Appendix}
\addcontentsline{toc}{section}{Appendix}

Let us recall some basic facts of order theory.

\begin{definition}
    Let $X\neq\emptyset$, a binary relation $R$ on $X$ is a subset of $X\times X$. We write 

    \begin{equation}
        xRy \text{ if } \{x,y\}\in R.
    \end{equation}
\end{definition}

We define some basic properties of a binary relation.

\begin{definition}
    Let $R$ be a binary relation on $X$ non-empty. We define

\begin{itemize}
        \item Reflexivity: $\forall x\in X:\, xRx$.
        \item Irreflexivity: $\nexists x\in X:\, xRx$.
        \item Antisymmetry: $xRy,yRx \Rightarrow x=y$.
        \item Symmetry: $xRy \Rightarrow yRx$.
        \item Transitivity: $xRy,yRz \Rightarrow xRz$.
        \item Completness: $\forall x,y\in X \Rightarrow xRy\text{ or }yRx$.
    \end{itemize}
\end{definition}

Typically, a preorder is a binary relation that satisfies reflexivity, transitivity, and symmetry, and a set with a preorder is called a poset. A linear order is a complete preorder, and a set with a linear order is called a loset. 
Most of order theory is developed on these concepts. Instead, we use antisymmetry instead of symmetry, and irreflexivity instead of reflexivity. 


Let us now define the transitive closure of a binary relation.

\begin{definition}
    Let $R$ be a binary relation on $X$ non-empty. The transitive closure of $R$ is the smallest transitive binary relation that contains $R$. It is denoted with $T(R)$.
\end{definition}

Inclusion of binary relation is intended in the canonical set inclusion, that is given $R,S$ binary relation we say that $R$ is included in $S$ if

\[xRy \Rightarrow xSy,\]

or equivalently $R\subseteq S$.

\begin{proposition}
    Every binary relation $R$ has a transitive closure $T(R)$.

    \begin{proof}
        The binary relation $S=X\times X$ is transitive and contains $R$. Furthermore, the transitivity of a binary relation is closed under intersection. Indeed, if $V,W$ are transitive binary relations then $U=V\cap W$ is transitive: if $xUy,yUz$ then $xVy,yVz$ and the same thing for $W$, then by transitivity of $V,W$ we have $xUz$ and $xVz$, that is $xUz$. 
    \end{proof}
\end{proposition}

We define the extension of a binary relation.

\begin{definition}
    Let $R$ be a binary relation on $X$ non-empty. An extension $R^{\ast}$ of $R$ is a binary relation that contains $R$ and preserves the properties of $R$.  
\end{definition}

We now arrive at the fundamental result used in the jungle policies section.

\begin{theorem*}
    Let $X$ non-empty and $R$ irreflexive, transitive, and antisymmetric binary relation, then there exists a complete extension of $R$.
\end{theorem*}

Before proving the theorem we clarify an unproved result that we will use, the Hausdorff Maximum principle. Let us assume the Axiom of Choice.

\begin{axiom}
    Let $\mathcal{A}$ be an arbitrary collection of non-empty sets. Then there exists a function $f:\mathcal{A} \rightarrow \cup_{A\in\mathcal{A}}A$ such that

    \[f(A)\in A,\,\forall A\in\mathcal{A}.\]
\end{axiom}

It can be proven that the axiom of choice is equivalent to the Hausdorff's Maximum principle. A proof of necessity can be found in Rudin \cite{rudin1976principles} (p. 395).

\begin{axiom}
    % Every set $X$ with an irreflexive, transitive and antisymmetric binary relation admits a maximal subset $Y$ with an irreflexive, transitive and antisymmetric binary relation.
    There exists an $\supseteq$-maximal loset in every poset.
\end{axiom}

Maximality is intended as no other subset with such a binary relation contains $Y$ and is in $X$. Thanks to this result we can prove the theorem, assuming the Axiom of Chioce.

\begin{proof}
    Let $T_X$ be the set of all extensions of $R$. By definition of set inclusion $\supseteq$ then $(T_X,\supseteq)$ is a poset. Let $(A,\supseteq)$ be a maximal loset in $T_X$. 
    Let us define $R^{\ast}=\cup_{S\in A}S$, that is the union of all binary relations in $A$. Since $A$ is a loset, then union of all its element is its maximal element, therefore $R^{\ast}$ 
    is an element of $A$, and therefore an extension of $R$. We now show that $R^{\ast}$ is complete. If it wasn't there'd be at least two elements $x,y\in X$ unranked by $R^{\ast}$. 
    Let us define $R'=T\left(R^{\ast}\cup\{x,y\}\right)$. It is an extension of $R$\footnote{It can be directly proven.} and strictly contains $R^{\ast}$, which contradicts the maximality of $A$.
\end{proof}

\end{document}